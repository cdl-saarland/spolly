% Chapter 4

\chapter{Evaluation} % Write in your own chapter title
\label{Chapter4}
\lhead{Chapter 4. \emph{Evaluation}} % Write in your own chapter title to set the page header

\red
\begin{shaded}

The evaluation, as essential part of this thesis, allows to compare the this 
work with others which share the same goals. Since it is hard to put effort in
figures, the evaluation may provide information about the results of them.
Correctness, applicability and speedup have been the stated goals which were 
tested using the SPEC 2000 and Polybench 3.2 benchmark suites. While the former
one contains general benchmarks used to evaluate most new optimizations, the 
later one is especially designed for polyhedral optimizations as done by SPolly.
These benchmarks might not reflect the reality in everyday optimization but
they might be used to compare SPolly and Polly 
(as presented in the diploma thesis of Tobias Grosser \cite{grosser:thesis}).


\section{The Environment}

(more details in table \ref{tab:EvaluationEnvironment}). 
\begin{wraptable}[]{r}{0.42\textwidth}
  \caption{The evaluation environment}
  \begin{center}
    \begin{tabular}{ r | c c }
      & A & B \\
      \hline
            CPU & i5 M560 & X5570 \\ 
    clock speed & 2.67GHz & 2.93GHz \\
    smart cache & 3MB & 8MB \\
        \#cores & 2 & 8 \\
      \#threads & 4 & 16 \\
            RAM & 6GB & 24 GB \\
           LLVM & 3.0 debug & 3.0 \\
             OS & Arch  & Gentoo R7 \\
    \end{tabular}
  \end{center}
  \label{tab:EvaluationEnvironment}
\end{wraptable}
running Arch linux with an
\textit{Intel(R) Core(TM) i5 CPU M 560 @ 2.67GHz} and 6GB RAM. Parallel versions
could use up to four simultaneous running threads. 
The second one ... TODO ... 
As the compile time evaluation is machine independent, it was 
performed on the general purpose machine only. Contrary, the runtime evaluation 
has been performed twice, once on each machine. 

The work and thus the evaluation is based on an LLVM 3.0 build 
with enabled assertions and disabled optimization. All source files have been 
converted by clang to LLVM-IR files, optimized by opt and ... TODO linked TODO

\begin{center}TODO picture of the chain\end{center}


\section{Compile Time Evaluation}
The main part of the compile time evaluation aims to get quantitative results 
about valid and invalid sSCoPs. These results correspond with the applicability 
of this work, as they both outline how many regions can be taken into account now
and which work is needed to increase this number. 
As mentioned earlier this part is mainly machine independent, since the quantitative
results are. There is one case where compile time transformation can improve the
program with no need of speculation at all. 
These cases are explained and evaluated separately in section 
\ref{soundCTtransformations}.


\subsection{Preperation}
TODO
 -basicaa -indvars -mem2reg -polly-independent -polly-region-simplify -polly-prepare 


\subsection{Quantitative Results}

\subsubsection{SPEC2000}
TODO why not all spec2000 benchmarks ? \\
TODO 

\end{shaded}


\begin{figure}[htbp]
	\centering
        \includegraphics[width=0.9\textwidth]{SPEC2000CT.pdf}
		\rule{35em}{0.5pt}
	\caption{Numbers of valid and speculative valid SCoPs}
	\label{fig:SPEC2000CT}
\end{figure}

\subsubsection{Polybench 3.2}
\begin{figure}[htbp]
	\centering
        \includegraphics[width=0.9\textwidth]{polybenchCT.pdf}
		\rule{35em}{0.5pt}
	\caption{Numbers of valid and speculative valid SCoPs}
	\label{fig:polybenchCT}
\end{figure}

\begin{table}[htbp]
  \caption{Results of running Polly and SPolly on SPEC 2000 benchmarks}
  \begin{tabular}{| l | r | r | r | r | r | r | r |}
    \hline
    %\multirow{2}{*}{\textbf{Benchmark}} & \multirow{2}{*}{\textbf{\#func}} & \multicolumn{2}{c|}{\textbf{\#simple regions}} & \multirow{2}{*}{\textbf{\#instr}} &  \multicolumn{2}{c|}{\textbf{valid SCoPs}} &  \multicolumn{2}{c|}{\textbf{Avg detection time}} \\
    \multirow{2}{*}{\textbf{Benchmark}} & \multirow{2}{*}{\textbf{\#instr}} & \multicolumn{2}{c|}{\textbf{\#simple regions}} &  \multicolumn{2}{c|}{\textbf{valid SCoPs}} &  \multicolumn{2}{c|}{\textbf{Avg detec. time}} \\
    %\cline{6-9} \cline{3-4}
    \cline{3-8} 
    & & \textbf{initial} & \textbf{prepared} & \textbf{Polly} & \textbf{SPolly} & \textbf{Polly} & \textbf{SPolly} \\
    \hline
    \hline
    188.ammp   & 19824  & 205 & 208 & 12 & 45 & &  \\
    179.art    &  1667  & 66  &  66 &  5 & 16 & &  \\
    256.bzip2  &  3585  & 114 & 116 & 13 &  7 & &  \\
    186.crafty & 25541  & 305 & 310 & 23 & 23 & &  \\
    183.equake &  2585  &  70 &  71 & 10 & 15 & &  \\
    164.gzip   &  4773  &  92 &  95 &  6 &  0 & &  \\
    181.mcf    &  1663 &  33 &  33  &  0 &  0 & &  \\
    177.mesa   & 80952 & 816 & 832  & 94 &247 & &  \\
    300.twolf  & 35796 & 679 & 716  &  6 &  0 & &  \\
    175.vpr    & 19547 & 319 & 329  & 17 & 33 & &  \\
    \hline
  \end{tabular}
\end{table}
TODO numbers can be produced via /home/johannes/git/sambamba/testdata/spec2000/timeAnalysis.py


\red
\begin{shaded}
\subsubsection{Available tests}

\paragraph{Alias tests}


\paragraph{Invariant tests}




\subsection{Sound Transformations}
\label{soundCTtransformations}
As described earlier, region speculation collects violations within a SCoP 
and can introduce tests for some of them. There are cases when these tests will
suffice to get a sound result, thus there is no need for a runtime system at all.
Although this hold in respect to the soundness of a program, this does not mean 
performance will rise when these transformations are used.  \\
TODO scores -- heuristic / statistics 


\section{Runtime Evaluation}


\section{Problems}
During the work with LLVM 3.0 and a corresponding version of Polly a few
problems occurred. Some of them could not be reproduced in newer versions
they were just be tackled with tentative fixes, as they will be resolved as soon
as Sambamba and SPolly will be ported to a newer version.
Others, which could be reproduced in the current trunk versions,
have been reported and listed in figure \ref{tab:bugreports}. All bugs were 
reported with a minimal test case and a detailed description why they occur.

\end{shaded}

% ID | description | status | patch included | tool
\begin{table}[htbp]
  \caption{Reported bugs}
  \begin{tabularx}{0.9\textwidth}{ c | X | p{2cm} | c | c }
   ID & Description & Status & Patch provided  & Component \\
  \hline \hline
  12426 & Wrong argument mapping in OpenMP subfunctions & RESOLVED FIXED & yes & Polly \\
   \hline
  12427 & Invariant instruction use in OpenMP subfunctions & NEW & yes & Polly \\
   \hline
  12428 & PHInode use in OpenMP subfunctions & NEW & no & Polly \\
   \hline
  12489 & Speed up SCoP detection & NEW & yes & Polly \\
  TODO & add the others to bugzilla & & & \\
  TODO & PollybenchC2 gemver , 2mm & & & \\
  \end{tabularx}
  \label{tab:bugreports}
\end{table}
