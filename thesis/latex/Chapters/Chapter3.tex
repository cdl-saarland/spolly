% Chapter 3

\chapter{Concept} % Write in your own chapter title
\label{Chapter3}
\lhead{Chapter 3. \emph{Concept}} % Write in your own chapter title to set the page header

From a high point of view, this work tries to weaken the harsh requirements on 
SCoPs in order to make Pollys loop optimizations applicable on a wider range of
programs. Apart from the implementation work, which will be described in the 
next chapter, immense effort has been made on the concepts and key ideas behind.
We believe that these ideas and the knowledge gained during the work is very 
valuable not only for future work on SPolly or one of its bases but also for
other approaches facing similar situations. On the way to a working version 
many pitfalls have been encountered that should be avoided in the future, 
perhaps with similar approaches we worked out. 


\section{SPolly In A Nutshell}


\section{Region Scores}

\lstset{frame=none}
\begin{wrapfigure}[]{r}{0.4\textwidth}
  \centering
  \subfloat[complete static sSCoP]{
    \lstinputlisting{Primitives/Code/sSCoPstatic.c}
    \label{lst:sSCoPstatic}  
  }

  \subfloat[Branch within a sSCoP]{
    \lstinputlisting{Primitives/Code/sSCoPbranch.c}
    \label{lst:sSCoPbranch}  
  }

  \subfloat[irreversible call within a sSCoP]{
    \lstinputlisting{Primitives/Code/sSCoPprintf.c}
    \label{lst:sSCoPprintf}  
  }
  \caption{example sSCoPs}
  \label{fig:ScoredSCoPs}
\end{wrapfigure}
\resetlst

Region scores are the heuristic used to decide whether or not a sSCoP is worth
to speculate on, thus for which region should profiling and parallel versions
be created and used. As the former ones may change the score again it is 
reasonable to create parallel version later if the collected data suggest to do
so. Initial efforts to create these scores did not use any kind of 
memory, thus every call needed to reconsider the whole region. To avoid this
unnecessary computations the current score is a symbolic value which may contain
variables for variables not known statically, branch probabilities and the 
introduced tests. Evaluation of these symbolic values will take all profiling
information into account and yield a comparable integer value. 
Only during the initial score creation region speculation will
traverse the region to find parameters and branches for later annotation.
All instructions will be scored and the violating ones
will be checked for their speculative potential.
As memory instructions are 
guarded by the STM for the case the speculation failed, calls may not be
reversible, thus without any speculative potential at all. Such function calls
are not checked earlier since the region speculation needs the information about
possible other branches within this region. Listing \ref{lst:sSCoPprintf} 
provides such an example but these cases will be revisited in the next 
two chapters too. Table \ref{tab:Scores} lists the scores for the examples in 
figure \ref{fig:ScoredSCoPs}. 


\begin{table}[htbp]
  \centering
  \caption{Scores for the sSCoPs presented in various listings}
  \begin{tabular}{ c l}
    listing & score \\
    \hline
    \ref{lst:sSCoPstatic} & $ 576 \hfill \text{  (if \texttt{A,B} and \texttt{C} may alias)} $ \\
    \ref{lst:sSCoPbranch} & $63 * (11 + ((7 * \text{@if.then\_ex\_prob}) / 100) + ((5 * \text{@if.else\_ex\_prob}) / 100)) $ \\
    \ref{lst:sSCoPprintf} & $((0\text{ smax }\%\text{N}) / 16) * (6 + (-1000 * \text{@if.then\_ex\_prob} / 100)$ \\
   \end{tabular}
  \label{tab:Scores}
\end{table}



%\section{}


\clearpage
\begin{figure}[htbp]
  \centering
  \includegraphics[width=0.6\textwidth]{Figures/draftPaperCT.eps}
  \caption{Draft paper: SPolly at compile time}
  \label{fig:draftPaperCT}  
\end{figure}
\begin{figure}[htbp]
  \centering
  \includegraphics[width=0.6\textwidth]{Figures/draftPaperRT.eps}
  \caption{Draft paper: SPolly at runtime}
  \label{fig:draftPaperCT}  
\end{figure}
\clearpage

