% Chapter 2

\chapter{Background Theory} % Write in your own chapter title
\label{Chapter2}
\lhead{Chapter 2. \emph{Background Theory}} % Write in your own chapter title to set the page header

TODO ...  which will be now introduced. 
As an introduction will not cover the whole theory and all capabilities 
further readings are listed. As these should not be needed to relate 

\begin{center}
TODO this section was just copied TODO \\
TODO it is also incomplete TODO 
\end{center}




\section{LLVM - The Low Level Virtual Machine}
\label{LLVM}
The Low Level Virtual Machine is a compiler infrastructure designed to optimize
during compiletime, linktime and runtime. Originally designed for C/C++, 
many other frontends for a variety of languages exist by now. The source is 
translated into an intermediate representation (LLVM-IR), which is available 
in three different, but equivalent forms. There is the in-memory compiler IR, 
the on-disk bitcode representation and human readable assembly language.
The LLVM-IR is a type-safe, static single assignment based language,
designed for low-level operations. It is capable of 
representing high-level structures in a flexible way.
Due to the fact that LLVM is built in a modular 
way and can be extended easily, most of the state of the art analysis and
optimization techniques are implemented and shipped with LLVM. Plenty of other
extensions, e.g., Polly, can be added by hand. 
Another point for LLVM is the 
included just-in-time compiler for runtime analysis and optimization mentioned
in the context of Sambamba in Section \ref{sambamba} in more detail.

\subsection*{Further Reading}

\begin{itemize}
  \item A Compilation Framework for Lifelong Program Analysis \& Transformation
    \cite{LLVM:CGO04}  
  \item \url{http://www.llvm.org} \nocite{LLVM:Online}
\end{itemize}

% =============================================================================


\section{The Polyhedral Model}
The polyhedral model is a mathematical description of a, may be infinite set of 
integer points defined by a finite system of affine inequalities. 
In the finite case they are also called polytopes.
Polyhedra can be used to describe the index space of a loop nest with 
affine loop bounds. The dimension of this space is determined by the depth of 
the loop nest. Each occurring iteration vector, which is a vector 
containing all surrounding induction variables, corresponds to a point in the 
index space. In addition to the index space the polytope model consists of a 
set of vectors within this index space. Each vector corresponds to a data
dependency between two iterations.

Figure \ref{fig:loopInThePolytopeModel} shows such a loop represantation in the 
polytope model. The iterations are shown as points and
the dependencies between two iterations are represented as arrows. 
For simplicity the initialization process of the array is skipped.

\subsection*{Further Reading}
\begin{itemize}
  \item Loop Parallelization in the Polytope Model \cite{Lengauer93loopparallelization}  
  \item PoCC - The Polyhedral Compiler Collection \cite{PoCC:Online}
\end{itemize}


\section{Polly - A Polyhedral Optimizer For LLVM}
\label{Polly}
As parallelism and data locality becomes more and more important, 
there is heavy research and development in this area.
%, to catch up with the increasing count of processors in nowadays machines
Based on the polyhedral 
model, Polly aims for an automatic optimization of
LLVM bitcode with no need for annotations or other user interactions.
Besides the support of external optimizers, there is a 
state of the art polyhedral library included, as well as support for SIMD and 
OpenMP code generation\cite{raghesh2011framework}. 

To gain a deeper understanding,  the three main parts of Polly, 
represented as (solid) translations in Figure \ref{fig:ArchitectureOfPolly},
are explained in particular now.

\subsection*{Further Reading}

\begin{itemize}
  \item Polly - Polyhedral optimization in LLVM \cite{grosser.11.impact}  
  \item Enabling Polyhedral Optimizations in LLVM \cite{grosser:thesis}
  \item A Framework for Automatic OpenMP Code Generation \cite{raghesh2011framework}
  \item \url{http://polly.llvm.org} \nocite{Polly:Online}
\end{itemize}


\section{Sambamba - A Framework For Adaptive Program Optimization}
As an extension for LLVM the \textit{Sambamba} compiler framework is designed to
allow runtime analyses and (speculative) optimization.
Furthermore these optimization can create and refine runtime profiles which
are used to recalibrate and specialize the (speculative) execution. Method 
versioning allows conservative and speculative 
versions of a method to be stored and switched during runtime. 
%But both can profit from specialization at runtime. 
%especially after conflicting speculation.
%Based on the LLVM suite, Sambamba uses the shipped JIT compiler 
%and a software transactional memory system to secure
%speculative execution. 
Written in a completely modular way, Sambamba extensions consist 
of a static part (compiletime) and a dynamic one (runtime). 
Both extension parts can use Sambamba to store information,
collected at the corresponding time, accessible for the dynamic part at 
runtime.
%In the context of speculation and profiling, Sambamba can be used to store
%several version of a method which can be generated by one of the two parts.
Profiling combined with the method versioning system allows runtime interactions
to explore more parallelism or minimize the overhead in case of misspeculation. 


\subsection*{Further Reading}
\begin{itemize}
  \item Sambamba: A Runtime System for Online Adaptive Parallelization \cite{DBLP:conf/cc/StreitHZH12}  
  \item \url{http://www.sambamba.org} \nocite{StreitHZH12:Online}
\end{itemize}

