% Chapter 6

\chapter{Case Study} % Write in your own chapter title
\label{Chapter6}
\lhead{Chapter 6. \emph{Case Study}} % Write in your own chapter title to set the page header


\section{Matrix Multiplication}
\label{MatrixMultiplication}
Matrix multiplication is a well known computational problem and part of many 
algorithms. If the data size grows, the runtime may have crucial impact on th
e overall performance. Tiling, vectorization and parallel execution yield an
enormous speedups as different approaches already
showed\cite{grosser:thesis, JIMBOREAN-2012-664345},
still the question about applicability remains. A slightly modified source code 
would not be optimized at all, even if the computation has not been changed. 

This section will compare different implementations of a simple 2d
matrix multiplication. As sample size  $1024*1024$ floats are used 
(\texttt{N} is defined as 1024).
Each example is executed 10 times and the geometric mean of the results 
(without the best and worst one) is computed. All numbers are generated on 
the server described in table \ref{todo}. The base algorithm stays the same 
for each case, so there is no hand made optimization involved. The computed 
result is checked each time in order to prevent false optimizations.


\subsection{Case A}
\begin{wrapfigure}[]{r}{0.5\textwidth}
  \centering
    \begin{minipage}[c]{0.4\textwidth}
    \vspace*{-3mm}
    \lstinputlisting{Primitives/Code/matmul1prep.c}
    \end{minipage}
  \caption{Matmul case A}
   \label{lst:MatmulVersionA}
\end{wrapfigure}

Listing \ref{lst:MatmulVersionA} shows the matrix multiplication as used in 
many presentations and benchmarks. This case is quite grateful because the
global arrays are distinct and fixed in size. Furthermore the loop nest is
perfectly nested and all memory accesses can be computed statically.
With this in mind the popularity of this case is hardly surprising,
just as the outstanding results are.\\


\subsection{Case B}
\begin{wrapfigure}[]{l}{0.5\textwidth}
    \begin{minipage}[c]{0.4\textwidth}
    \vspace*{-3mm}
    \lstinputlisting{Primitives/Code/matmul2prep.c}
    \end{minipage}
    \caption{Matmul case B}
    \label{lst:MatmulVersionB}
\end{wrapfigure}

Case B (see listing \ref{lst:MatmulVersionB}) is very similar to the previous one.
The arrays are still fixed in size but now given as arguments. As the declaration 
is not global anymore, aliasing between \texttt{A,B} and \texttt{C} is possible. 
Nevertheless are the memory accesses for all three pointers still computable.
\\


\subsection{Case C and D}
\begin{figure}[htpb]
  \centering
  \subfloat[Matmul case C] {
    \begin{minipage}[c][5cm]{0.4\textwidth}
    \lstinputlisting{Primitives/Code/matmul3prep.c}
    \end{minipage}
   \label{lst:MatmulVersionC}
  }
  \subfloat[Matmul case D] {
    \begin{minipage}[c][5cm]{0.4\textwidth}
    \lstinputlisting{Primitives/Code/matmul4prep.c}
    \end{minipage}
    \label{lst:MatmulVersionD}
  }
  \caption{Matmul case C and D}
   \label{lst:MatmulVersionCD}
\end{figure}
Listing \ref{lst:MatmulVersionA} shows the matrix multiplication as used in 
many presentations and benchmarks. This case is quite grateful because the
global arrays are distinct, the loop nest is perfectly nested and all 
memory accesses can be computed statically. With this in mind the popularity of
this case is hardly surprising, just as the good results of all tested versions 
are. 


Listing \ref{lst:MatmulVersionA} shows the matrix multiplication as used in 
many presentations and benchmarks. This case is quite grateful because the
global arrays are distinct, the loop nest is perfectly nested and all 
memory accesses can be computed statically. With this in mind the popularity of
this case is hardly surprising, just as the good results of all tested versions 
are. 

\begin{table}[htpb]

  \caption{Case study results}
  \label{tab:CaseStudyResults}
\end{table}


%\begin{figure}[htpb]
  %\centering
  %\subfloat[Matmul version 1]{

  %} \hfill
  %\subfloat[Matmul version 2]{
    %\begin{minipage}[c]{0.45\textwidth}
    %\lstinputlisting{Primitives/Code/matmul2prep.c}
    %\label{lst:MatmulVersion2}
    %\end{minipage}
  %}

  %\subfloat[Matmul version 3]{
    %\begin{minipage}[c]{0.45\textwidth}
    %\lstinputlisting{Primitives/Code/matmul3prep.c}
    %\label{lst:MatmulVersion3}
    %\end{minipage}
  %}
  %\hfill
  %\subfloat[Matmul version 4]{
    %\begin{minipage}[c]{0.45\textwidth}
    %\lstinputlisting{Primitives/Code/matmul4prep.c}
    %\label{lst:MatmulVersion4}
    %\end{minipage}
  %}
  %\caption{Matrix multiplication in different versions }
%\end{figure}
