% Appendix A

\chapter{Polly -- Optimization Options}
\label{AppendixB}
\lhead{Appendix B. \emph{Polly -- Optimization Options}}

Different options may yield different results. Since this simple truth holds for
Polly too, the options, listend in the Table below, may yield improvements even
for already well performing SCoPs.
\\

\begin{table}[htbp]
  \begin{framed}
  \caption{Overview about the optimization options of Polly}
  \begin{tabularx}{\textwidth}{ l p{1mm} X}
    Option         && Description \\
    \hline
    -polly-no-tiling              && Disable tiling in the scheduler  \\
    -polly-tile-size=N\footnotemark[1] && Create tiles of size N \\
    -polly-opt-optimize-only=STR  && Only a certain kind of dependences (all/raw) \\
    -polly-opt-simplify-deps      && Simplify dependences within a SCoP    \\
    -polly-opt-max-constant-term  && The maximal constant term allowed (in the scheduling) \\
    -polly-opt-max-coefficient    && The maximal coefficient allowed (in the scheduling)  \\
    -polly-opt-fusion             && The fusion strategy to choose (min/max) \\
    -polly-opt-maximize-bands     && Maximize the band depth (yes/no) \\
    -polly-vector-width=N\footnotemark[1]  && Try to create vector loops with N iterations \\
    -enable-polly-openmp          && Enable OpenMP parallelized loop creation \\
    -enable-polly-vector          && Enable loop vectorization (SIMD) \\
    -enable-polly-atLeastOnce     && Indicates that every loop is at leas executed once \\
    -enable-polly-aligned         && Always assumed aligned memory accesses \\
    -enable-polly-grouped-unroll  && Perform grouped unrolling, but don't generate SIMD \\
     &&  \\ 
  \end{tabularx}
  \label{tab:PollyOptions}
  \end{framed}
\end{table}


\vfill
\vfill
\vfill
\rule{\textwidth}{0.1mm}
  \footnotemark[1] Not available from the command line \\

\clearpage

\chapter{SPolly as Static Optimizer}
\label{AppendixC}
\lhead{Appendix C. \emph{SPolly As Static Optimizer}}
Table \ref{tab:CommandLineOptions} lists which non-speculative functionalities of
SPolly can be used without the presence of Sambamba. 
The new code generation options can be used exactly the same as the OpenMP and vecorizer backend options.
However, the ``-spolly'' options are exclusively allowed with ``-polly-detect''.
Optimizations and parallel code 
generation are explicitly done by the region speculation, and the opt tool may
break if they are used from the outside, too.
\\

\begin{table}[htbp]
  \begin{framed}
  \caption{Command line options to interact with SPolly}
  \begin{tabularx}{\textwidth}{ l p{1mm} X  }
    Command line option     && Description  \\
    \hline
    -enable-spolly          && Enables SPolly during SCoP detection,\par
                                (options containing ``spolly'' will not work without) \\ 
    -spolly-replace         && Replaces all sSCoPs by optimized versions \par 
                                (may not be sound)  \\
    -spolly-replace-sound   && As spolly-replace, but sound due to runtime checks \par
                                (no effect if checks are not sound)\\
    -spolly-extract-regions && Extracts all sSCoPs into their own sub function \\
    -polly-forks=N          && Set the block size which is used when 
                             polly-fork-join code generation is enabled\\
    -enable-polly-fork-join && Extracts the body of the outermost,
                             parallelizeable loop, performs loop blocking with
                             block size N and unrolls the new introduced loop 
                             completely \par
                             (one loop with N calls in the body remains)\\
    -polly-inline-forks     && Inline the call instruction in each fork \\
  \end{tabularx}
  \label{tab:CommandLineOptions}
  \end{framed}
\end{table}

